\subsection{E91}
\begin{frame}
	\frametitle{E91}
	-- Entwickelt durch Artur Ekert im Jahr 1991\\
	-- Verwendung von Bell States \\
	-- Auswahl zufälliger Polarisationsbasen\\
	-- Bob nutzt die inverse Interpretation von Alice \\
		\hspace{0.5em} (unter Verwendung von $|\mathnormal{\Psi}^{(-)}\rangle$) \\
	\vspace{1em}
	\hspace{0.5em}
	\begin{tabular}{c|c|c|c|c}
		Basis & H & V & D & A \\
		\hline
		Alice & 0 & 1 & 0 & 1 \\
		\hline
		Bob & 1 & 0 & 1 & 0 \\
	\end{tabular}
\end{frame}

\begin{frame}
	\frametitle{Mathematischer Beweis$^{1}$ I}
	\begin{align}
		|\mathnormal{\Psi}^{(-)}\rangle = \frac{1}{\sqrt{2}} \left\{|H\rangle_{A}|V\rangle_{B} - |V\rangle_{A}|H\rangle_{B}\right\}
	\end{align}
	Übergang zur diagonalen Basis:
	\begin{align*}
		|H\rangle_A = a_H^+|0\rangle = \frac{1}{\sqrt{2}}(a_D^++a_A^+)|0\rangle = \frac{1}{\sqrt{2}}(|D\rangle_A+|A\rangle_A) \\
		|V\rangle_A = a_H^+|0\rangle = \frac{1}{\sqrt{2}}(a_D^+-a_A^+)|0\rangle = \frac{1}{\sqrt{2}}(|D\rangle_A-|A\rangle_A)
	\end{align*}
	\footnotetext[1]{Aus \cite{Chekhova_Lecture4_12} {\"ubernommen}}
\end{frame}

\begin{frame}
	\frametitle{Mathematischer Beweis$^{1}$ II}
	\framesubtitle{Photon B}
	\begin{alignat*}{2}
		|\mathnormal{\Psi}^{(-)}\rangle & = \frac{1}{2\sqrt{2}}\left\{(|D\rangle_A+|A\rangle_A)(|D\rangle_B-|A\rangle_B) - (|D\rangle_A+|A\rangle_A)(|D\rangle_B-|A\rangle_B)\right\}\\
		& = \frac{1}{2\sqrt{2}}\left\{|A\rangle_A|D\rangle_B - |D\rangle_A|A\rangle_B)\right\}
	\end{alignat*}
	
	% Um zu überprüfen, ob jemand mithört, testen Alice und Bob die Bellschen Ungleichungen.
	\footnotetext[1]{Aus \cite{Chekhova_Lecture4_12} {\"ubernommen}}
\end{frame}
